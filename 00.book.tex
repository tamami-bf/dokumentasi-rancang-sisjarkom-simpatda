\documentclass[pdftex,12pt, oneside]{article}

%\usepackage[paperwidth=8.5in, paperheight=13in]{geometry} % Folio
\usepackage[paperwidth=8.27in, paperheight=11.69in]{geometry} % A4

\usepackage{makeidx}         % allows index generation
\usepackage{graphicx}        % standard LaTeX graphics tool
                             % when including figure files
\usepackage[bottom]{footmisc}% places footnotes at page bottom
\usepackage[english]{babel}
\usepackage{enumerate}
\usepackage{paralist}
\usepackage{float}
\usepackage{gensymb}  
\usepackage{listings}
\usepackage{color}
\usepackage{mathtools} % atau \usepackage{amsmath}
\renewcommand{\baselinestretch}{1.5}

\newcommand{\HRule}{\rule{\linewidth}{0.5mm}}

\definecolor{codegreen}{rgb}{0,0.6,0}
\definecolor{codegray}{rgb}{0.5,0.5,0.5}
\definecolor{codepurple}{rgb}{0.58,0,0.82}
\definecolor{backcolor}{rgb}{0.95,0.95,0.92}

\lstdefinestyle{mystyle}{
  backgroundcolor=\color{backcolor},
  commentstyle=\color{codegreen},
  keywordstyle=\color{magenta},
  stringstyle=\color{codepurple},
  basicstyle=\footnotesize,
  breakatwhitespace=false,
  breaklines=true,
  captionpos=b,
  keepspaces=true,
  numbers=left,
  numbersep=5pt,
  showspaces=false,
  showstringspaces=false,
  showtabs=false,
  tabsize=2
}

\lstset{style=mystyle}


\begin{document}
\sloppy % biar section ga melebar melewati kertas

\begin{center}
{\large DOKUMENTASI RANCANGAN SISTEM JARINGAN KOMPUTER - SIMPATDA}
\\[1cm]
xx Januari 2017\\
Priyanto Tamami, S.Kom.
\end{center}

%\frontmatter%%%%%%%%%%%%%%%%%%%%%%%%%%%%%%%%%%%%%%%%%%%%%%%%%%%%%%


%%%%%%%%%%%%%%%%%%%%%%%%%%%%%%%%%%%%%%%%%%%%%%%%%%%%%%%%%%%%%%%%%%%%%%

\section{ANALISIS KEBUTUHAN}

\subsection{PENDAHULUAN}

Dengan dilakukannya perubahan struktur kedinasan, maka dibutuhkan penyesuaian terhadap struktur jaringan komputer, terutama dalam pelaksanaan 

\subsection{METODE ANALISIS KEBUTUHAN}

\subsection{ANALISA DAN SIMPULAN}

\section{STUDI KELAYAKAN}

\subsection{MASALAH}

\subsection{LINGKUNGAN IMPLEMENTASI}

\subsection{REKOMENDASI}

\subsection{KONFIGURASI SISTEM ALTERNATIF}

\subsection{RUANG LINGKUP SISTEM}

\subsection{KELAYAKAN SISTEM}

\subsection{SIMPULAN}

\section{METODE KOMUNIKASI}

\section{TOPOLOGI JARINGAN}

\section{DIAGRAM JARINGAN}

\section{PERANGKAT KERAS DAN PERANGKAT LUNAK JARINGAN}


\end{document}